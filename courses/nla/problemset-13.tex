\documentclass{problemset}

\usepackage{nicematrix}

\author{Michael van Straten}
\Lecture{Numerische Lineare Algebra}
\Problemset{13}
\DoOn{3. Februar 2025, 09:00 Uhr}

\begin{document}

\maketitle

\setcounter{problem}{2}

\begin{problem}[Gerschgorin-Kreise]{5 Punkte}
Sei \( n \in \nats \) und \( A \in \mathbb{C}^{n \times n} \). Dann heißt für
\( i \in \{1, \ldots, n\} \) die Menge
\[
    K_i := \left\{ z \in \mathbb{C} : || \leq \sum_{j=1, j \neq i}^n |a_{ij}| \right\}
\]
der \( i \)-te Gerschgorin-Kreis von \( A \). Zeigen Sie: Ist \( \lambda \in
  \mathbb{C} \) ein Eigenwert von \( A \), so gilt
\[
    \lambda \in \bigcup_{i=1}^n K_i.
\]
\begin{proof}
    Sei \((\lambda, v)\) ein Eigenpaar von \(A\) sowie \(i\) sodass \(|v_i|\)
    maximal ist. Dann folgt, dass \newcommand{\idx}{\substack{j=1 \\ j \neq i}}
    \begin{align*}
        Av = \lambda v & \iff \sum_{j=1}^n a_{ij} v_j = \lambda v_i                    \\
                       & \iff \sum_{\idx}^n a_{ij} v_j = \lambda v_i - a_{ii} v_i      \\
                       & \iff \sum_{\idx}^n a_{ij} v_j = (\lambda - a_{ii}) v_i        \\
                       & \iff \lambda - a_{ii} = \sum_{\idx}^n a_{ij} \frac{v_j}{v_i}.
    \end{align*}

    Daraus folgt
    \begin{align*}
        \abs{\lambda - a_{ii}}
         & = \big| \sum_{\idx}^n a_{ij} \frac{v_j}{v_i} \big| \\
         & \le \sum_{\idx}^n \abs{a_{ij}},
    \end{align*}
    was die Behauptung zeigt.
\end{proof}
\end{problem}

\begin{problem}[Vorkonditionierung mit unvollständiger Cholesky-Zerlegung]{8 Punkte}
Sei
\[
    A :=
    \begin{bmatrix}
        2  & -1 & -1 & -1 \\
        -1 & 2  & 0  & 0  \\
        -1 & 0  & 2  & 0  \\
        -1 & 0  & 0  & 2
    \end{bmatrix}, \quad
    b :=
    \begin{bmatrix}
        -1 \\ 1 \\ 1 \\ 1
    \end{bmatrix}, \quad
    x_0 :=
    \begin{bmatrix}
        0 \\ 0 \\ 0 \\ 0
    \end{bmatrix}.
\]
Berechnen Sie:
\begin{enumerate}
    \item mittels Aufgabe 3 eine Schätzung für den kleinsten und größten
          Eigenwert von \( A \),
    \item zwei Schritte des relaxierten Richardson-Verfahrens zur Lösung von \(
          Ax = b \) und beginnend bei \( x_0 \). Verwenden Sie dabei \( \omega
          = 1 / \bar{\lambda} \), wobei \( \bar{\lambda} \) dem Mittelwert der
          beiden Schätzungen aus (i) entspricht,
    \item eine unvollständige Cholesky-Zerlegung \( \tilde{L} \tilde{L}^\top
          \approx A \) und \( \|A - \tilde{L} \tilde{L}^\top \|_\infty \),
    \item die vorkonditionierte Matrix \( \tilde{A} := \tilde{L}^{-1} A
          \tilde{L}^{-\top} \),
    \item mittels Aufgabe 3 eine Schätzung für den kleinsten und größten
          Eigenwert von \( \tilde{A} \),
    \item zwei Schritte des vorkonditionierten relaxierten
          Richardson-Verfahrens zur Lösung von \( Ax = b \) und beginnend bei
          \( x_0 \). Verwenden Sie dabei \( \omega = 1 / \bar{\lambda} \),
          wobei \( \bar{\lambda} \) dem Mittelwert der beiden Schätzungen aus
          (v) entspricht.
\end{enumerate}
\end{problem}

\begin{problem}[Modellierung des max-flow Problems]{7 Punkte}
In einer technischen Anlage soll eine möglichst große Menge einer Flüssigkeit
von einem Startpunkt zu einem Zielpunkt transportiert werden. Dazu steht ein
Netzwerk von Rohren zur Verfügung. Die Rohre sind an Knotenpunkten verbunden.
An jedem Knotenpunkt lässt sich der Durchfluss zu den anliegenden Rohren mit
Hilfe von Kompressoren steuern. Für jedes Rohr ist vorab eine Flussrichtung
festgelegt und ein maximal möglicher Durchfluss spezifiziert. In den
Knotenpunkten kann keine Flüssigkeit zwischengespeichert werden.

Modellieren Sie die Situation als lineares Optimierungsproblem in den
Standardformen (P) und (N) aus der Vorlesung. Geben Sie dazu in beiden Fällen
die Dimensionen \( n, m \in \nats \), sowie die Vektoren \( c \in \reals^n \),
\( b \in \reals^m \) und die Matrix \( A \in \reals^{m \times n} \) an.

\textbf{Lösung:}

Für die Knotenpunkte \( V_1, \ldots, V_n \), wobei \( V_1 \) den Start- und \(
V_n \) den Endpunkt des Netzwerks definiert, sowie \( E_1, \ldots, E_m \),
welche die Rohre zwischen den einzelnen Knotenpunkten repräsentieren,
definieren wir die Systemmatrix \( A \) wie folgt:
\[
    A \coloneq
    \renewcommand{\arraystretch}{1.5}
    \begin{array}{c}
        V_1 \\ V_2 \\ \vdots \\ V_m \\ \hline E_1 \\ \vdots \\ E_n
    \end{array}
    \overset{
        \begin{array}{llll}
            E_{\text{opt}} & E_1 & \ldots & E_m
        \end{array}
    }
    {
        \begin{pmatrix}
            1      & d_{1, 1} & \cdots & d_{1, m} \\
            0      & d_{2, 1} & \cdots & d_{2, m} \\
            \vdots & \vdots   & \cdots & \vdots   \\
            -1     & d_{n, 1} & \cdots & d_{n, m} \\
            \hline
            0      &          &        &          \\
            \vdots &          & I_m    &          \\
            0      &          &        &          \\
        \end{pmatrix}
    }
    \in \reals^{(n+m) \times (m+1)},
\]
wobei
\[
    d_{ij} \coloneq \begin{cases}
        1  & \text{wenn \( V_i \) von \( E_j \) gespeist wird}, \\
        -1 & \text{wenn \( V_i \) von \( E_j \) abläuft},       \\
        0  & \text{sonst}.
    \end{cases}
\]
Das Rohr \( E_{\text{opt}} \) verbindet den Startpunkt \( V_1 \) und den
  Endpunkt \( V_n \) direkt miteinander. Unsere Optimierung zielt auf dieses
  Rohr ab, daher lautet der Optimierungsvektor:
\[
    c = e_1 \in \reals^{m+1},
\]
wobei \( e_1 \) der Einheitsvektor ist.

Für die rechte Seite \( b \) definieren wir:
\[
    b \coloneq \begin{pmatrix}
        0 \\ \vdots \\ 0 \\ \hline t_1 \\ \vdots \\ t_m
    \end{pmatrix} \in \reals^{m+1},
\]
wobei \( t_i \) der maximale Durchfluss des Rohrs \( E_i \) ist.

\textbf{Standardform (P):}
\[
    \max \quad c^\top x \quad \text{unter den Nebenbedingungen} \quad A x \leq b, \; x \geq 0.
\]

\textbf{Normalform (N):}
\[
    \min \quad -c^\top x \quad \text{unter den Nebenbedingungen} \quad A x = b, \; x \geq 0.
\]

\textbf{Interpretation:}
Die Variable \( x \in \mathbb{R}^{m+1} \) enthält den maximalen Durchfluss des
Rohrs \( E_{\text{opt}} \) in der ersten Komponente und die optimalen
Durchflussraten in den Komponenten \( 2 \) bis \( m+1 \).
\end{problem}

\end{document}
