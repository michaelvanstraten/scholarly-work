\documentclass{problemset}

\author{Michael van Straten}
\Lecture{Numerische Lineare Algebra}
\Problemset{11}
\DoOn{20. Januar 2025}

\begin{document}

\maketitle

\setcounter{problem}{2}

\begin{problem}{8 Punkte}

Sei \( n \in \nats \). Zeigen Sie, dass es zu jedem \( \epsilon > 0 \) und
jeder Matrix \( A \in \reals^{n \times n} \), deren charakteristisches Polynom
über \( \reals \) in Linearfaktoren zerfällt, eine induzierte Matrixnorm gibt,
für die gilt:
\[
    \|A\| \leq \rho(A) + \epsilon.
\]
\textit{Hinweis:}
Betrachten Sie für eine Transformationsmatrix \( S \in \reals^{n \times n} \),
welche \( A \) in Jordan-Normalform überführt, und die Diagonalmatrix \( D \in
\reals^{n \times n} \) mit den Einträgen \( (1, \epsilon, \dots,
\epsilon^{n-1}) \) die Norm
\[
    \|x\| := \|D^{-1} S^{-1} x\|_\infty.
\]
\begin{proof}
    Sei \( A \in \reals^{n \times n} \) mit Jordan-Normalform \( J \in
    \reals^{n \times n} \), bestehend aus \( k \) Jordan-Blöcken \( J_1,
    \ldots, J_k \), und Transformationsmatrix \( S \), sodass \( A = S J S^{-1}
    \). Sei \( l_1, \ldots, l_k \) die Größen der Jordan-Blöcke, also \( J_m
    \in \reals^{l_m \times l_m} \) für \( m = 1, \ldots, k \), wobei \(
    \sum_{m=1}^k l_m = n \). Wählen wir die Norm \( \|x\| \) wie im Hinweis
    definiert, dann gilt:
    \begin{align*}
        \|A\| & = \sup_{x \neq 0} \frac{\|Ax\|}{\|x\|}                                                     \\
              & = \sup_{x \neq 0} \frac{\|D^{-1} S^{-1} A x\|_\infty}{\|D^{-1} S^{-1} x\|_\infty}          \\
              & = \sup_{x \neq 0} \frac{\|D^{-1} S^{-1} S J S^{-1} x\|_\infty}{\|D^{-1} S^{-1} x\|_\infty} \\
              & = \sup_{x \neq 0} \frac{\|D^{-1} J S^{-1} x\|_\infty}{\|D^{-1} S^{-1} x\|_\infty}          \\
              & = \sup_{x \neq 0} \frac{\|D^{-1} J D D^{-1} S^{-1} x\|_\infty}{\|D^{-1} S^{-1} x\|_\infty} \\
              & \leq \|D^{-1} J D\|_\infty.
    \end{align*}

    Nun betrachten wir die Matrix \( D^{-1} J D \). Diese ergibt sich als:
    \[
        D^{-1} J D =
        \begin{bmatrix}
            1 &               &        &                   \\
              & \epsilon^{-1} &        &                   \\
              &               & \ddots &                   \\
              &               &        & \epsilon^{-(n-1)}
        \end{bmatrix}
        \begin{bmatrix}
            J_1 &     &        &     \\
                & J_2 &        &     \\
                &     & \ddots &     \\
                &     &        & J_k
        \end{bmatrix}
        \begin{bmatrix}
            1 &          &        &                \\
              & \epsilon &        &                \\
              &          & \ddots &                \\
              &          &        & \epsilon^{n-1}
        \end{bmatrix}.
    \]

    Definieren wir
    \[
        d \coloneq \sum_{i = 1}^{m-1} l_i,
    \]
    so ergibt sich für den \( m \)-ten Jordan-Block \( J_m \) der Größe \( l_m
    \):
    \[
        \begin{bmatrix}
            \epsilon^{-d} &                 &        &                         \\
                          & \epsilon^{-d-1} &        &                         \\
                          &                 & \ddots &                         \\
                          &                 &        & \epsilon^{-d - l_m + 1}
        \end{bmatrix}
        \begin{bmatrix}
            \lambda_m & 1         &        &           \\
                      & \lambda_m & 1      &           \\
                      &           & \ddots & 1         \\
                      &           &        & \lambda_m
        \end{bmatrix}
        \begin{bmatrix}
            \epsilon^{d} &                  &        &                        \\
                         & \epsilon^{d + 1} &        &                        \\
                         &                  & \ddots &                        \\
                         &                  &        & \epsilon^{d + l_m - 1}
        \end{bmatrix}
        =
        \begin{bmatrix}
            \lambda_m & \epsilon  &          &           \\
                      & \lambda_m & \epsilon &           \\
                      &           & \ddots   & \epsilon  \\
                      &           &          & \lambda_m
        \end{bmatrix}.
    \]

    Die Skalierung durch \( D^{-1} \) und \( D \) reduziert die Größe der
    Nebendiagonalelemente um den Faktor \( \epsilon \). Daher gilt für die
    Spaltensummennorm von \( D^{-1} J D \):
    \[
        \|D^{-1} J D\|_\infty \leq \rho(A) + \epsilon,
    \]
    wobei \( \rho(A) \) den Spektralradius von \( A \) bezeichnet. Damit ist
    die Behauptung gezeigt.
\end{proof}
\end{problem}

\begin{problem}[Das Richardson-Verfahren für SPD-Matrizen]{6 Punkte}
Sei \( n \in \mathbb{N} \). Ermitteln Sie für jede symmetrisch positiv definite
Matrix \( A \in \reals^{n \times n} \) die Werte des Parameters \( \omega
\in \reals \), sodass die lineare Verfahrensfunktion \( \Phi: \reals^n
\times \reals^n \to \reals^n \), gegeben durch
\[
    \Phi(x, b) = (I - \omega A)x + \omega b,
\]
konvergent und konsistent zu \( A \) ist.
\end{problem}

\begin{problem}[Lemma 6.8]{6 Punkte}
Sei \( n \in \mathbb{N} \), \( A \in \reals^{n \times n} \). Weiter seien
\( \Phi \) und \( \Psi \) zwei lineare Verfahrensfunktionen mit zugehörigen
Matrizen \( B_\Phi, B_\Psi, C_\Phi \) und \( C_\Psi \) der ersten Normalform.
Mit \( \Phi \circ \Psi \) sei die Produktiteration
\[
    (\Phi \circ \Psi)(x, b) := \Phi(\Psi(x, b), b)
\]
bezeichnet. Zeigen Sie:
\begin{enumerate}
    \item Sind \( \Phi \) und \( \Psi \) konsistent zu \( A \), dann ist auch
          die Produktiteration \( \Phi \circ \Psi \) konsistent zu \( A \).
    \item Sind \( \Phi \) und \( \Psi \) konvergent, dann ist auch die
          Produktiteration \( \Phi \circ \Psi \) konvergent.
    \item Bezeichnen \( B_{\Phi \circ \Psi} \) bzw. \( B_{\Psi \circ \Phi} \)
          die entsprechenden Matrizen der ersten Normalformen der
          Produktiterationen \( \Phi \circ \Psi \) bzw. \( \Psi \circ \Phi \),
          dann gilt:
          \[
              \rho(B_{\Phi \circ \Psi}) = \rho(B_{\Psi \circ \Phi}).
          \]
\end{enumerate}
\begin{proof}\leavevmode
    \begin{enumerate}
        \item Sei \( x \) die Lösung des Gleichungssystems \( A x = b \). Dann
              folgt für die Produktiteration aufgrund der Konsistenz von \(
              \Phi \) und \( \Psi \):
              \begin{align*}
                  (\Phi \circ \Psi)(x, b) & = \Phi(\Psi(x, b), b) \\
                                          & = \Phi(x, b)          \\
                                          & = x.
              \end{align*}
              Somit ist auch \( \Phi \circ \Psi \) konsistent zu \( A \).

        \item Wir beobachten für die Produktiteration:
              \begin{align*}
                  (\Phi \circ \Psi)(x, b) & = \Phi(\Psi(x, b), b)                          \\
                                          & = B_\Phi(B_\Psi x + C_\Psi b) + C_\Phi b       \\
                                          & = B_\Phi B_\Psi x + B_\Phi C_\Psi b + C_\Phi b \\
                                          & = B_\Phi B_\Psi x + (B_\Phi C_\Psi + C_\Phi) b \\
                                          & = \underbrace{B_\Phi B_\Psi}_{\eqcolon B} x
                  + \underbrace{(B_\Phi C_\Psi + C_\Phi)}_{\eqcolon C} b.
              \end{align*}
              Für die Norm von \( B \) gilt:
              \[
                  \|B\| = \|B_\Phi B_\Psi\| \leq \|B_\Phi\| \|B_\Psi\| \leq 1 \cdot 1 = 1.
              \]
              Nach Theorem 6.9 folgt damit, dass die Produktiteration \( \Phi
              \circ \Psi \) konvergent ist.

        \item Betrachten wir zwei Produktmatrizen \( AB \) und \( BA \), so
              haben diese denselben Spektralradius. Dies folgt aus:
              \[
                  ABv = \lambda v \quad \Rightarrow \quad (BA)(Bv) = \lambda (Bv).
              \]
              Daraus folgt, dass die Eigenwerte von \( AB \) und \( BA \)
              identisch sind. Insbesondere haben \( B_{\Phi \circ \Psi} \) und
              \( B_{\Psi \circ \Phi} \) denselben maximalen Eigenwert:
              \[
                  \rho(B_{\Phi \circ \Psi}) = \rho(B_{\Psi \circ \Phi}).
              \]
    \end{enumerate}
\end{proof}
\end{problem}

\end{document}
