\documentclass{problemset}

\Lecture{Mathematik}
\Problemset{1}
\DoOn{28. Oktober 2024}
\author{Michael van Straten}

\setlist[enumerate, 1]{label=(\alph*)}

\begin{document}

\maketitle

\setcounter{problem}{3}

\begin{problem}{4 Punkte}
\begin{enumerate}
    \item Gegeben sei die Matrix
          \begin{equation*}
              A =
              \begin{bmatrix}
                  7.2 & 0.4 \\
                  4.6 & 7.2 \\
              \end{bmatrix}.
          \end{equation*}
          Bestimmen Sie die Spektralnorm von \( A \).

    \item Bestimmen Sie die Spektralnorm einer orthogonalen Matrix \(Q \in
          \mathbb{R}^{n \times n}\).
\end{enumerate}
\end{problem}

\begin{problem}{6 Punkte}
Zeigen Sie: Für die Normen \( \| \cdot \|_X = \| \cdot \|_1 \) auf \(
\mathbb{R}^n \) und \( \| \cdot \|_Y = \| \cdot \|_1 \) auf \( \mathbb{R}^m \)
gilt für Matrizen \( A \in \mathbb{R}^{m \times n} \):
\begin{equation*}
    \norm{A}_1 \coloneq \sup_{\norm{x}_1 = 1} \norm{A x}_1
    = \max_{j=1, \ldots, n} \sum_{i=1}^{m} \abs{a_{ij}}.
\end{equation*}
Diese Norm wird daher für Matrizen auch Spaltensummennorm genannt.
\end{problem}

\begin{problem}{10 Punkte}
Sei \( A \in \mathbb{R}^{m \times n} \) eine Matrix, \( \| \cdot \|_X \) eine
Norm auf \( \mathbb{R}^n \) und \( \| \cdot \|_Y \) eine Norm auf \(
\mathbb{R}^m \). Zeigen Sie, dass

\begin{equation*}
    \sup_{\norm{x}_X = 1} \norm{A x}_Y
    = \sup_{\norm{x}_X \leq 1} \norm{A x}_Y
    = \sup_{x \in \mathbb{R}^n \setminus \{0\}} \frac{\norm{A x}_Y}{\norm{x}_X}
    = \inf \{ c > 0 : \norm{A x}_Y \leq c \norm{x}_X \, \forall x \in \mathbb{R}^n \}.
\end{equation*}
Damit können alle vier Ausdrücke als Definition von \( \norm{A}_{Y,X} \)
verwendet werden.

\begin{proof}
    Zunächst nehmen wir an, dass
    \begin{equation*}
        \norm{A}_{X, Y} \ge \sup_{\norm{x}_X \le 1} \norm{A x}_Y.
    \end{equation*}
    Da \(\set{x : \norm{x}_X = 1} \subseteq \set{x : \norm{x}_X \le 1}\), folgt daraus
    \begin{equation*}
        \norm{A}_{X, Y} \le \sup_{\norm{x}_X \le 1} \norm{A x}_Y.
    \end{equation*}
    Somit müssen diese Terme nach dem Squeeze-Theorem gleich sein:
    \begin{equation*}
        \norm{A}_{X, Y} = \sup_{\norm{x}_X \le 1} \norm{A x}_Y.
    \end{equation*}

    Nun nehmen wir an,
    \begin{equation*}
        \norm{A}_{X, Y} \le \sup_{\norm{x}_X \le 1} \norm{A x}_Y,
    \end{equation*}
    und sei \(\norm{\tilde{x}}_X < 1\). Nach Aufgabe 1.1 gilt dann:
    \begin{equation}\label{1}
        \norm{A \tilde{x}}_Y \le \norm{A}_{X, Y} \norm{\tilde{x}}_X.
    \end{equation}

    Da \(\norm{\tilde{x}}_X < 1\), folgt für alle \(\tilde{x}\) dieser Form,
    dass \(\norm{A}_{X, Y} \ge \norm{A \tilde{x}}_Y\) gilt. Daraus folgt erneut
    die Gleichheit der beiden Terme nach dem Squeeze-Theorem.

    Setzen wir nun \(A \neq 0\), \(a = \norm{A}_{X,Y}\), und
    \begin{equation*}
        b = \inf \set{c > 0 : \norm{A x}_Y \leq c \norm{x}_X \, \forall x \in \mathbb{R}^n}.
    \end{equation*}
    Angenommen \(a \le b\), so folgt aus der positiven Definitheit von
    \(\norm{\cdot}_Y\), dass \(a > 0\) ist. Da \eqref{1} für \(a\) gilt,
    erhalten wir
    \begin{equation*}
        a \le b \le a \Rightarrow a = b.
    \end{equation*}

    Angenommen, \(a \ge b\). Da \(a > 0\) und \eqref{1} für \(a\) gilt, folgt
    wiederum
    \begin{equation*}
        b \ge a \Rightarrow a = b.
    \end{equation*}

    Schließlich folgt die Gleichheit von
    \begin{equation*}
        \sup_{\norm{x}_X \leq 1} \norm{A x}_Y \quad
        \text{und} \quad \sup_{x \in \mathbb{R}^n \setminus \{0\}} \frac{\norm{A x}_Y}{\norm{x}_X}
    \end{equation*}
    direkt aus der Gleichheit der zugrunde liegenden Mengen.
\end{proof}
\end{problem}

\end{document}
