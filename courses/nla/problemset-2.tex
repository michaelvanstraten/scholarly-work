\documentclass{problemset}

\author{Michael van Straten}
\Lecture{Numerische Lineare Algebra}
\Problemset{2}
\DoOn{4. November 2024}

\setlist[enumerate, 1]{label=(\alph*)}

\begin{document}

\maketitle

\setcounter{problem}{3}

\begin{problem}{7 Punkte}
Gegeben seien

\[
    A =
    \begin{bmatrix}
        2 & 2  & 1  & 2 \\
        1 & 2  & 0  & 2 \\
        6 & 6  & 1  & 8 \\
        7 & 10 & 10 & 5 \\
    \end{bmatrix}, \quad
    b_1 =
    \begin{bmatrix}
        3  \\
        21 \\
        -1 \\
        41 \\
    \end{bmatrix}, \quad
    b_2 =
    \begin{bmatrix}
        2  \\
        3  \\
        -6 \\
        3  \\
    \end{bmatrix}.
\]

Bestimmen Sie mithilfe der LR-Zerlegung (ohne Pivotisierung) die Lösungen der
Gleichungssysteme \(Ax = b_i\) für \(i = 1, 2\). Wie verändert sich der Aufwand
der Rechenoperationen zum Lösen des zweiten Systems, wenn das erste System
bereits mit der LR-Zerlegung gelöst wurde?
\end{problem}

\begin{problem}{6 Punkte}
Gegeben sei die Matrix

\[
    A = \begin{bmatrix} \tilde{A} & v \\ u^T & a \end{bmatrix} \in \mathbb{R}^{(n+1) \times (n+1)},
\]

wobei \(\tilde{A} \in \mathbb{R}^{n \times n}\) eine reguläre Matrix ist, deren
LR-Zerlegung bekannt ist, und \(u, v \in \mathbb{R}^n\) sowie \(a \in
\mathbb{R}\).
\begin{enumerate}
    \item Bestimmen Sie eine LR-Zerlegung von \(A\) mithilfe der LR-Zerlegung
          von \(\tilde{A}\).
    \item Zeigen Sie: \(A\) ist regulär \(\iff a - u^T \tilde{A}^{-1} v \neq
          0\).
\end{enumerate}
\begin{proof}
    \leavevmode
    \begin{enumerate}
        \item \label{problem:5:a} Aus der Blockmatrix-Multiplikation und der
              Invertierbarkeit von \( \tilde{A} \) ergibt sich:
              \begin{align*}
                  A & = \begin{bmatrix}
                            1                  & 0 \\
                            u^T \tilde{A}^{-1} & 1
                        \end{bmatrix}
                  \begin{bmatrix}
                      \tilde{A} & v                        \\
                      0         & a - u^T \tilde{A}^{-1} v
                  \end{bmatrix}                  \\
                    & = \begin{bmatrix}
                            1                  & 0 \\
                            u^T \tilde{A}^{-1} & 1
                        \end{bmatrix}
                  \begin{bmatrix}
                      \tilde{A} & 0 \\
                      0         & 1
                  \end{bmatrix}
                  \begin{bmatrix}
                      1 & \tilde{A}^{-1} v         \\
                      0 & a - u^T \tilde{A}^{-1} v
                  \end{bmatrix}                          \\
                    & = \begin{bmatrix}
                            1                  & 0 \\
                            u^T \tilde{A}^{-1} & 1
                        \end{bmatrix}
                  \begin{bmatrix}
                      L R & 0 \\
                      0   & 1
                  \end{bmatrix}
                  \begin{bmatrix}
                      1 & \tilde{A}^{-1} v         \\
                      0 & a - u^T \tilde{A}^{-1} v
                  \end{bmatrix}                          \\
                    & = \begin{bmatrix}
                            1                  & 0 \\
                            u^T \tilde{A}^{-1} & 1
                        \end{bmatrix}
                  \begin{bmatrix}
                      L & 0 \\
                      0 & 1
                  \end{bmatrix}
                  \begin{bmatrix}
                      R & 0 \\
                      0 & 1
                  \end{bmatrix}
                  \begin{bmatrix}
                      1 & \tilde{A}^{-1} v         \\
                      0 & a - u^T \tilde{A}^{-1} v
                  \end{bmatrix}                          \\
                    & = \underbrace{\begin{bmatrix}
                                            L                    & 0 \\
                                            u^T \tilde{A}^{-1} L & 1
                                        \end{bmatrix}}_{\eqcolon \tilde{L}}
                  \underbrace{
                      \begin{bmatrix}
                          R & R \tilde{A}^{-1} v       \\
                          0 & a - u^T \tilde{A}^{-1} v
                      \end{bmatrix}}_{\eqcolon \tilde{R}}.
              \end{align*}

              Da \( L \) eine untere Dreiecksmatrix und \( R \) eine obere
              Dreiecksmatrix sind, sind auch \( \tilde{L} \) und \( \tilde{R}
              \) entsprechend untere und obere Dreiecksmatrizen.

        \item Wir beweisen nun die Aussage in beiden Richtungen:
              \begin{itemize}
                  \item[\(\Rightarrow\)]
                        Angenommen, \( A \) ist regulär und \( A = \tilde{L}
                        \tilde{R} \), wobei \( \tilde{L} \) und \( \tilde{R} \)
                        wie in \ref{problem:5:a} definiert sind. Dann müssen
                        auch \( \tilde{L} \) und \( \tilde{R} \) in \(
                        \mathbb{R}^{(n+1) \times (n+1)} \) regulär sein. Da \(
                        R \) regulär und eine obere Dreiecksmatrix ist, folgt:
                        \begin{equation*}
                            \prod_{i=1}^n (\tilde{R})_{ii} \neq 0,
                        \end{equation*}
                        was impliziert, dass \( a - u^T \tilde{A}^{-1} v \neq 0 \).

                  \item[\(\Leftarrow\)]
                        Angenommen, \( a - u^T \tilde{A}^{-1} v \neq 0 \). Da
                        \( \tilde{A} \) regulär ist, folgt, dass \( R \)
                        regulär ist und somit auch \( \tilde{R} \) regulär sein
                        muss. Die Regulärität von \( \tilde{L} \) und damit auch von \(
                        A \) folgt daraus, dass die Diagonalelemente in \(
                        \tilde{L} \)
                        alle den Wert 1 haben.
              \end{itemize}
    \end{enumerate}
\end{proof}
\end{problem}

\begin{problem}{7 Punkte}
Gegeben sei die Tridiagonalmatrix

\[
    A =
    \begin{bmatrix}
        a_1    & c_1    & 0      & \dots   & 0       \\
        b_1    & a_2    & c_2    & \ddots  & \vdots  \\
        0      & b_2    & a_3    & \ddots  & 0       \\
        \vdots & \ddots & \ddots & \ddots  & c_{n-1} \\
        0      & \dots  & 0      & b_{n-1} & a_n     \\
    \end{bmatrix} \in \mathbb{R}^{n \times n}.
\]

Leiten Sie für den Fall, dass die LR-Zerlegung \(A = LR\) mit

\[
    L =
    \begin{bmatrix}
        1       & 0      & \dots   & 0      \\
        d_1     & 1      & \ddots  & \vdots \\
        \vdots  & \ddots & \ddots  & 0      \\
        d_{n-1} & \dots  & d_{n-2} & 1      \\
    \end{bmatrix}, \quad
    R =
    \begin{bmatrix}
        e_1    & f_1    & 0      & \dots   & 0       \\
        0      & e_2    & f_2    & \ddots  & \vdots  \\
        \vdots & \ddots & \ddots & \ddots  & 0       \\
        0      & \dots  & 0      & e_{n-1} & f_{n-1} \\
        0      & \dots  & \dots  & 0       & e_n     \\
    \end{bmatrix}
\]

existiert, Rekursionsformeln für \(e_i\) (\(i = 1, \ldots, n\)) und \(d_i\),
\(f_i\) (\(i = 1, \ldots, n-1\)) her. Welchen Aufwand hat die Berechnung der
LR-Zerlegung mittels dieser Rekursion?
\end{problem}

\end{document}
