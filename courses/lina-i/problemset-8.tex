\documentclass{problemset}

\Lecture{Lineare Algebra und Analytische Geometrie I}
\Problemset{8}
\DoOn{10. Dezember 2023}
\author{Michael van Straten}

\begin{document}
\maketitle

\begin{problem}[Basen endlichdimensionaler Vektorräume]{3 Punkte}
Sei $n \in \mathbb{N}$, $V$ ein $K$-Vektorraum mit $\dim_K(V) = n$ und $B =
    \{v_1, \ldots, v_n\} \subseteq V$. Zeigen Sie, dass die folgenden Aussagen
äquivalent sind:
\begin{enumerate}
    \item $B$ ist eine Basis.
    \item $B$ ist linear unabhängig.
    \item $B$ ist ein Erzeugendensystem.
\end{enumerate}

\end{problem}

\begin{problem}[Vektoren mit 0 an der i-ten Stelle]{3 Punkte}
Sei $K \in \{ \mathbb{R}, \mathbb{C} \}$ und $n \in \mathbb{N}$ mit $n > 1$.
Weiter seien $\bar{e}_i := (1, \ldots, 1, 0, 1, \ldots, 1) \in K^n$, $i \in
    \{1, \ldots, n\}$ die Vektoren mit 0 an der $i$-ten Stelle und 1 sonst.
Berechnen Sie $\dim_K(\operatorname{Span}\{\bar{e}_i \mid i \in \{1, \ldots,
    n\}\})$.
\end{problem}

\begin{problem}[Unterraum von Polynomen]{6 Punkte}
Bezeichne $R\leq 3[t]$ den Vektorraum der Polynome über $\mathbb{R}$ mit Grad
höchstens drei. Sei weiter $U := \{p(t) \in R\leq 3[t] \mid p(0) = p(1) = 0\}$.
\begin{enumerate}
    \item Zeigen Sie, dass $U$ ein Unterraum von $R\leq 3[t]$ ist.
    \item Bestimmen Sie eine Basis von $U$ und berechnen Sie
          $\dim_{\mathbb{R}}(U)$.
    \item Geben Sie Polynome an, welche die in ii) angegebene Basis von $U$ zu
          einer Basis von $R\leq 3[t]$ erweitern.
\end{enumerate}
\end{problem}

\begin{problem}[Endliche Körper und Vektorräume]{8 Punkte}
\begin{enumerate}
    \item Sei $(L, +, \cdot)$ ein Körper und $K \subseteq L$ ein Unterkörper.
          Zeigen Sie, dass $(L, +, \cdot)$ ein $K$-Vektorraum ist.

    \item Betrachten Sie die Menge $\mathbb{Z}_4 := \{0, 1, 2, 3\}$ sowie
          $\operatorname{mod}_4 : \mathbb{Z} \to \mathbb{Z}_4$, $z \mapsto y$
          mit $\{y\} = [z] \equiv 4 \cap \mathbb{Z}_4$, das heißt
          $\operatorname{mod}_4(z)$ wählt den (eindeutigen) Repräsentanten von
          $[z] \equiv 4$ aus der in $\mathbb{Z}_4$ ist. Seien $+_4 :
          \mathbb{Z}_4 \times \mathbb{Z}_4 \to \mathbb{Z}_4$ und $\cdot_4 :
          \mathbb{Z}_4 \times \mathbb{Z}_4 \to \mathbb{Z}_4$ definiert durch
          \[ x +_4 y := \operatorname{mod}_4(x + y), \quad x \cdot_4 y := \operatorname{mod}_4(x \cdot y). \]
          Zeigen Sie, dass $(\mathbb{Z}_4, +_4, \cdot_4)$ kein Körper ist.

    \item Sei $F_4 := \{0, 1, a, b\}$ und seien $+_F4 : F_4 \times F_4 \to F_4$
          und $\cdot_F4 : F_4 \times F_4 \to F_4$ definiert durch:
          \[ \begin{array}{c|cccc}
                  +_{F_4} & 0 & 1 & a & b \\
                  \hline
                  0       & 0 & 1 & a & b \\
                  1       & 1 & 0 & b & a \\
                  a       & a & b & 0 & 1 \\
                  b       & b & a & 1 & 0 \\
              \end{array} \quad
              \begin{array}{c|cccc}
                  \cdot_{F_4} & 0 & 1 & a & b \\
                  \hline
                  0           & 0 & 0 & 0 & 0 \\
                  1           & 0 & 1 & a & b \\
                  a           & 0 & a & b & 1 \\
                  b           & 0 & b & 1 & a \\
              \end{array} \]
          Sei weiter $F_2 := \{0, 1\} \subseteq F_4$. Sie können ohne Beweis
          annehmen, dass $(F_4, +_{F_4}, \cdot_{F_4})$ ein Körper ist und $F_2$
          ein Unterkörper von $F_4$ ist. Geben Sie eine Basis von $F_4$ als
          Vektorraum über $F_2$ an und berechnen Sie $\dim_{F_2}(F_4)$.
    \item Berechnen Sie alle Nullstellen des Polynoms $X^2 + X + 1$ über $F_2$
          und $F_4$.
\end{enumerate}
\end{problem}

\end{document}
