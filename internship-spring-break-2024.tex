\documentclass{article}
\usepackage{enumitem}
\usepackage[german]{isodate}

\title{Praktikumsbericht}
\author{Michael van Straten}
\date{}

\begin{document}

\maketitle

\section*{Praktikumszeitraum:}
Wintersemesterpause 2024

\section*{Praktikumsposition:}
Studentische Hilfskraft im Bereich Storage Infrastructure

\section*{Einleitung:}

Während der Wintersemesterpause hatte ich die Gelegenheit, meine Rolle als
studentische Hilfskraft bei Mozilla zu einem Vollzeitpraktikum auszubauen.
Meine Hauptaufgabe bestand darin, ein verbessertes Tracing-System zu entwerfen
und zu implementieren, das Fehler auf der Client-Seite von Firefox aggregiert,
bevor sie an die Telemetrieserver von Firefox übermittelt werden. Dieser
Bericht dokumentiert die Ereignisse und Aktivitäten während dieses Praktikums.

\section*{Ereignisse und Aktivitäten während des Praktikums}

\begin{enumerate}[label=\textbf{Woche \arabic*:}]
    \item \daterange{2024-02-19}{2024-02-23}
          \begin{itemize}
              \item Ausführung der Fehler-Telemetrie-Analyse- und Berichtsskripte.
              \item Bearbeitung von Überprüfungskommentaren zu einer bestimmten Aufgabe.
              \item Einrichtung der kontinuierlichen Integration (CI) für den Aggregator.
          \end{itemize}

    \item \daterange{2024-02-26}{2024-03-01}
          \begin{itemize}
              \item Fortsetzung der Arbeiten am Aggregator und Recherche zur Graphimplementierung.
              \item Teilnahme an einem Treffen mit dem neuen CEO.
              \item Arbeit an der Bugzilla OpenAPI-Spezifikation und IPC-Dokumentation.
          \end{itemize}

    \item \daterange{2024-03-02}{2024-03-08}
          \begin{itemize}
              \item Diskussion zur Implementierung des verteilten Trace-Aggregators mit meinem
                    Mentor.
              \item Beginn der Arbeit am verteilten Trace-Aggregator und Teilnahme am Teamtreffen.
              \item Brainstorming zu Lösungen für die Implementierung.
              \item 1:1-Meeting mit meinem Vorgesetzten und Ausarbeitung von Ideen für einen Brief an den neuen CEO.
          \end{itemize}

    \item \daterange{2024-03-11}{2024-03-15}
          \begin{itemize}
              \item Arbeit an der IPC-Aggregation und Beginn der Arbeit an einer neuen
                    Tracing-Bibliothek.
              \item Fortsetzung des Lernens aus CPP-Vorträgen und Arbeit an der
                    IPC-Implementierung.
              \item Diskussion der Verwendung neuer Datapipes für die IPC-Aggregation.
          \end{itemize}

    \item \daterange{2024-03-18}{2024-03-22}
          \begin{itemize}
              \item Arbeit an Komponentenaggregation und Vorschlagsdokumenten.
              \item Gewährung von Commit-Level-3-Zugriff und Diskussion von Designähnlichkeiten mit
                    opentelemetry-cpp.
              \item Treffen mit meinem Mentor zur opentelemetry-cpp-Tracing-Pipeline.
          \end{itemize}

    \item \daterange{2024-03-25}{2024-03-29}
          \begin{itemize}
              \item Implementierung des IPC-Exporteurs und Arbeit an der internen
                    IPC-Implementierung.
              \item Fortsetzung der Implementierung der internen IPC-Implementierung und Arbeit an
                    weiteren internen IPC-Implementierungstypen.
          \end{itemize}

    \item \daterange{2024-04-01}{2024-04-05}
          \begin{itemize}
              \item Verwaltungsaufgaben und Erstellung des OpenTelemetry(C++)-Vorschlags.
          \end{itemize}

    \item \daterange{2024-04-08}{2024-04-12}
          \begin{itemize}
              \item Implementierung von Demopatches für den OpenTelemetry(C++)-Vorschlag.
              \item Teilnahme an Diskussionen auf GitHub im Zusammenhang mit Rust und Arbeit an der
                    Konfiguration des Systemaufbaus.
          \end{itemize}

\end{enumerate}

\section*{Fazit:}

Das Praktikum bei Mozilla während der Wintersemesterpause bot eine wertvolle
Gelegenheit, zur Entwicklung eines entscheidenden Bestandteils von Firefox
beizutragen. Durch Aufgaben von Design über Implementierung bis hin zur
Integration erlangte ich praktische Erfahrungen in
Softwareentwicklungspraktiken und Zusammenarbeit in einem professionellen
Umfeld. Dieses Praktikum hat meine Leidenschaft für Softwareentwicklung weiter
gestärkt und Einblicke vermittelt, die ich in meiner Karriere nutzen werde.

\section*{Hinweis:}

Die Überarbeitung dieses Berichts wurde mithilfe von ChatGPT durchgeführt, um
den Text zu verbessern und zu erweitern.

\end{document}
