\documentclass{exam}

\usepackage{amsthm}
\usepackage{amsmath}
\usepackage{amsfonts}

\title{Analysis I* - Aufgabenblatt 1}
\author{Michael van Straten}
\begin{document}
    \maketitle
    \section*{1. Summenformeln}
        Beweisen Sie folgende Summenformeln fur alle $n \in \mathbb{N}$:
        \begin{enumerate}
            \item[a)]
                \[
                    \sum_{k = 1}^{n} k^3 = \left(\sum_{k = 1}^{n}k\right)^2,
                \]
            \item[b)]
                \[
                    \sum_{k = 1}^{n-1} k^2(n-k)^2 = \frac{n(n^4-1)}{30}.
                \]
        \end{enumerate}
        \subsection*{Lösungen:}
        \begin{proof}[Bew.] 
            Für alle $n \in \mathbb{N}$ gilt $\sum_{k = 1}^{n} k^3 = \left(\sum_{k = 1}^{n}k\right)^2$.
            \begin{enumerate}
                \item[a)] \underline{Anfang}: $A(0)$ ist war: $Leere\ Summe = 0 = (Leere\ Summme)^2$
                \item[b)] \underline{Schritt}: Sei $n \ge n_0$. Angenommen $A(n)$ sei schon bewiesen.
                    Dann ist \begin{align}
                        \sum_{k = 1}^{n + 1} k^3 &= \sum_{k = 1}^{n}k^3 + (n + 1)^3 \\ 
                        &= \left(\sum_{k = 1}^{n}k\right)^2 + (n + 1)^3 \tag{Gausische summenformel} \\
                        &= \left(\frac{n(n+1)}{2}\right)^2 + (n + 1)^3 \\ 
                        &= \frac{n^2(n+1)^2}{4} + (n + 1)^3 \\
                        &= \frac{n^2(n+1)^2}{4} + \frac{4(n + 1)^3}{4} \\
                        &= \frac{n^2(n+1)^2 + 4(n + 1)^3}{4} \\
                        &= \frac{(n+1)^2(n^2 + 4(n + 1))}{4} \\
                        &= \frac{(n+1)^2(n^2 + 4n + 4))}{4} \\
                        &= \frac{(n+1)^2(n +2)^2}{4} \\
                        &= \left(\frac{(n+1)(n +2)}{2}\right)^2 \\
                        &= \left(\sum_{k=1}^{n+1}k\right)^2 
                    \end{align}
            \end{enumerate}
            Dann folgt aus a), b), dass $A(n)$ für alle $n \ge n_0$ wahr ist.
        \end{proof}
        \begin{proof}[Bew.]
            Für alle $n \in \mathbb{N}$ gilt $\sum_{k = 1}^{n - 1} k^2(n - k)^2 = \frac{n(n^4 - 1)}{30}$.
            \begin{enumerate}
                \item[a)] \underline{Anfang}: $A(1)$ ist war: $Leere\ Summe = \frac{1(1^4 - 1)}{30} = 0$ 
                \item[b)] \underline{Schritt}: Sei $n \ge n_0$. Angenommen $A(n)$ sei shon beweisen: Dann ist 
                    \begin{align}
                        \sum_{k = 1}^{n} k^2(n + 1 - k)^2 &= \frac{n((n+1)^4 - 1)}{30} + n^2(n+1-n)^2 \\
                        &= \frac{n((n+1)^4 - 1)}{30} + n^2(1)^2 \\
                        &= \frac{n((n+1)^4 - 1)}{30} + n^2 \\
                        &= \frac{n((n+1)^4 - 1)}{30} + \frac{30n^2}{30} \\
                        &= \frac{n((n+1)^4 - 1) + 30n^2}{30} \\
                    \end{align}
            \end{enumerate}
        \end{proof}
\end{document}
