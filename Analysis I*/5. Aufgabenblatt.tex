\documentclass{../problemset}

\Lecture{Analysis I} 
\Problemset{5} 
\DoOn{28.11.2023} 
\author{Michael van Straten}

\setlist[enumerate, 1]{label=\alph*)}
\setlist[enumerate, 2]{label=\alph*)}
\setlist[enumerate, 3]{label=\alph*)}
\setlist[enumerate, 4]{label=\alph*)}

\begin{document}
\maketitle

\begin{problem}[Aussagenlogik und Mengentheoreme]
Seien $A$, $B$, $C$ Mengen. Geben Sie, in Analogie mit Bsp Z2.4, detaillierte Beweise mithilfe der Aussagenlogik zu folgenden beiden Mengentheoremen (siehe Z1):
\begin{enumerate}
	\item $A \setminus (B \cup C) = (A \setminus B) \cap (A \setminus C)$
	\item $A \setminus (B \cup C) = (A \setminus B) \setminus C$
\end{enumerate}
\begin{proof}
	\begin{enumerate}
		\item $A \setminus (B \cup C) = (A \setminus B) \cap (A \setminus C)$

		      Sei $x$ ein beliebiges Element.
		      \begin{align*}
			      x \in A \setminus (B \cup C) & \Longrightarrow x \in A \land x \not \in (B \cup C)                             \\
			                                   & \Longrightarrow x \in A \land (x \not \in B \land x \not \in C)                 \\
			                                   & \Longrightarrow (x \in A \land x \not \in B) \land (x \in A \land x \not \in C) \\
			                                   & \Longrightarrow (x \in A \setminus B) \land (x \in A \setminus C)               \\
			                                   & \Longrightarrow x \in (A \setminus B) \cap (A \setminus C) \tag{\checkmark}     \\
		      \end{align*}

		\item $A \setminus (B \cup C) = (A \setminus B) \setminus C$

		      Sei $x$ ein beliebiges Element.
		      \begin{align*}
			      x \in A \setminus (B \cup C) & \Longrightarrow x \in A \land x \not\in (B \cup C)                 \\
			                                   & \Longrightarrow x \in A \land (x \not\in B \land x \not\in C)      \\
			                                   & \Longrightarrow x \in (A \setminus B) \land x \not\in C            \\
			                                   & \Longrightarrow x \in (A \setminus B) \setminus C \tag{\checkmark} \\
		      \end{align*}
	\end{enumerate}
\end{proof}
\end{problem}

\pagebreak

\begin{problem}[Abbildungen und Verknüpfungen]
\begin{enumerate}
	\item Seien $f: X \to Y$, $g: Y \to Z$ und $h = g \circ f$, also $h: X \to Z$.
	      \begin{enumerate}
		      \item Beweisen Sie, dass aus $h$ injektiv folgt, dass $f$ injektiv ist.
		      \item Beweisen Sie, dass $h$ surjektiv $\Rightarrow$ $g$ surjektiv.
		      \item Gilt $h$ surjektiv $\Rightarrow$ $f$ surjektiv? Gilt $h$ surjektiv $\Rightarrow$ $g$ injektiv? Wenn nicht, dann geben Sie Gegenbeispiele!
		      \item Seien nun noch vier Mengen $A$, $B$, $C$ und $D$ zusammen mit drei Abbildungen $f: A \to B$, $g: B \to C$ und $h: C \to D$ gegeben. Beweisen Sie, dass aus der Bijektivit¨at von $g \circ f$ und $h \circ g$ folgt, dass $f$, $g$ und $h$ alle bijektiv sind!
	      \end{enumerate}

\end{enumerate}

\begin{proof}
	\begin{enumerate}
		\item $h = g \circ f$ injektiv folgt, $f$ injektiv

		      Angenommen $f(x) = f(y)$ somit muss für $f$ injektiv gezeigt werden das $x = y$.

		      Betrachten wir $g$ angewendet auf $f(x), f(y)$, da $g$ eine Abbildung ist, muss Sie eindeutig sein somit gilt \[
			      g(f(x)) = g(f(y)),
		      \] da $h = g \circ f$ gilt \[
			      h(x) = h(y)
		      \], da $h$ injektiv muss somit aber $x = y$, was zu Zeigen war. Somit ist $f$ injektiv. \checkmark

		\item $h = g \circ f$ bijektiv folgt, $g$ bijektiv

		      Wir betrachten die zusammengesetzte Funktion \(g \circ f: X \to Z\) als eine surjektive Abbildung. Nach Definition bedeutet dies, dass für jedes \(z \in Z\) ein \(x \in X\) existiert, so dass \(g \circ f(x) = z\).

		      Wir faktorisieren nun \(g \circ f\) durch \(Y\):

		      \[ X \overset{f}{\to} Y \overset{g}{\to} Z; \]

		      Daher gilt für alle \(x \in X\), dass \(f(x) \in Y\). Somit existiert ein \(y = f(x) \in Y\) mit \(g(y) = g(f(x)) = g \circ f(x) = z\).

		      Da dies für jedes \(z \in Z\) gilt, ergibt sich:

		      \[ g: Y \to Z \]

		      Die Funktion \(g\) ist somit ebenfalls surjektiv.

		\item $h$ surjektiv $\not\Rightarrow$ $f$ surjektiv und $h$ surjektiv $\not\Rightarrow$ $g$ injektiv

		      Nehme $X = \set{x_1}, Y = \set{y_1, y_2}, Z=\set{z_1}$ sowie $f(x_1) = y_1, g(y_1) = g(y_2) = z_1$.
		      Somit ist $g \circ f$ surjektiv da für jedes $z \in Z$ ein $x \in X$ existiert, sodass $g \circ f(x) = z$.
		      $f$ ist hier allerdings nicht surjektiv da es für das Element $y_2 \in Y$ kein $x \in X$ gibt mit dem $f(x) = y_2$.
		      Somit ist bewiesen das aus $g \circ f$ surjektiv nicht unbedingt folgt, $f$ surjektiv. \checkmark

		      Da obere beispiel zeigt, auch das $g$ nicht injektiv sein muss da $g(y_1) = g(y_2)$ aber $y_1 \neq y_2$. \checkmark

		\item $g \circ f$ und $h \circ g$ bijektiv, folgt $f$, $g$ und $h$ bijektiv

		      Da $g \circ f$ bijektiv ist, ist $f$ injektiv und $g$ surjektiv.
		      Ebenso ist $g \circ h$ bijektiv, was bedeutet, dass $g$ injektiv und $h$ surjektiv ist.
		      Daher ist $g$ bijektiv.

		      Da $g$ bijektiv ist, existiert die inverse Funktion $g^{-1}$, die ebenfalls bijektiv ist.
		      Somit kann $f$ als die Komposition zweier bijektiver Funktionen geschrieben werden: $f = g^{-1} \circ (g \circ f)$.
		      Da die Komposition zweier bijektiver Funktionen wieder bijektiv ist, folgt, dass $f$ bijektiv ist.

		      Analog dazu lässt sich zeigen, dass auch $h$ bijektiv ist: $h = (h \circ g) \circ g^{-1}$.

	\end{enumerate}
\end{proof}
\end{problem}

\pagebreak

\begin{problem}[Graphen]
\begin{enumerate}
	\item Seien $X := \{0, 1, 2, 3, 4\}$ und $Y := \{0, 5, 10, 15\}$, sowie $A := X \times Y = \{(x, y) \mid x \in X, y \in Y\}$ und $B := \{x + y \mid x \in X, y \in Y\}$ gegeben. Definiere dann $R \subset A \times B$ durch $R := \{((x, y), x + y) \mid x \in X, y \in Y\}$.
	      Zeigen Sie, dass es eine bijektive Abbildung $f : A \to B$ gibt, so dass $R$ der Graph von $f$ ist.

	\item Sei $X$ eine nichtleere Menge. Welche Eigenschaften muss $X$ haben, damit $X \times X$ der Graph einer Abbildung von $X$ nach $X$ ist?

\end{enumerate}

\begin{proof}

	\begin{enumerate}
		\item Betrachten wir die Definition des Graphen einer Funktion $f$ \[
			      G(f) = \{(x,f(x)) : x \in X\},
		      \]

		      Definiere $f : A \rightarrow B; (x, y) \mapsto x + y$.
		      Somit ist lediglich zu zeigen das $f$ bijektiv, also injektiv und surjektiv, ist.

		      \textbf{$f$ surjektiv}:

		      Um zu Zeigen das $f$ surjektiv müssen wir zeigen das $\forall b \in B \Rightarrow \exists a \in A$ sodass $f(a) = b$.


	\end{enumerate}

\end{proof}

\end{problem}

\pagebreak

\begin{problem}[Injektive Abbildungen]
Seien $A$, $B$ nichtleere Mengen und $f : A \to B$, $g : B \to A$ injektive Abbildungen. Für jede Teilmenge $C \subset A$ sei die Menge $F(C)$ definiert durch
\[ F(C) := A \setminus g(B \setminus f(C)). \]

\begin{enumerate}
	\item Nehmen Sie zunächst an, es gebe eine nichtleere Teilmenge $C \subset A$ mit $F(C) = C$. Zeigen Sie, dass dann eine bijektive Abbildung von $A$ nach $B$ existiert.
	\item Lassen Sie nun die Zusatzannahme aus a) fallen, und beweisen Sie, dass solch eine Menge $C \subset A$ mit $F(C) = C$ existiert. (Dann haben Sie zusammen mit Ihrem Resultat aus a) bewiesen: Wenn jede von zwei Mengen $A$ und $B$ injektiv in die jeweils andere abgebildet werden kann, dann existiert eine Bijektion von $A$ auf $B$.)
\end{enumerate}

\begin{proof}
	\begin{enumerate}
		\item
	\end{enumerate}
\end{proof}
\end{problem}

\end{document}
