\documentclass{../problemset}

\Lecture{Analysis I}
\Problemset{6}
\DoOn{5.12.2023}
\author{Michael van Straten}

\begin{document}
\maketitle

\begin{problem}[Konvergenz und absolute Konvergenz]{10 Punkte}
\begin{enumerate}
	\item Untersuchen Sie folgende Reihe auf Konvergenz und absolute Konvergenz:
	      \[
		      \sum_{n=1}^{\infty} (-1)^n(\sqrt{n} + 1 - \sqrt{n}),
	      \]
	\item Untersuchen Sie folgende Reihe auf Konvergenz und absolute Konvergenz:
	      \[
		      \sum_{n=1}^{\infty} \frac{n^2}{3 + \frac{1}{n}\sqrt{n}},
	      \]
	\item Untersuchen Sie folgende Reihe auf Konvergenz und absolute Konvergenz:
	      \[
		      \sum_{n=2}^{\infty} \frac{2n+1}{5 \cdot 3^n},
	      \]
	\item Untersuchen Sie folgende Reihe auf Konvergenz und absolute Konvergenz:
	      \[
		      \sum_{n=1}^{\infty} \frac{2^n \cdot n!}{n^n},
	      \]
	\item Untersuchen Sie folgende Reihe auf Konvergenz und absolute Konvergenz:
	      \[
		      \sum_{n=0}^{\infty} \frac{n + 5}{n^2 - 4n + 1}.
	      \]
\end{enumerate}
\end{problem}

\begin{problem}[Binomialreihe]{6 Punkte}
\begin{enumerate}
	\item Sei $a \in \mathbb{R}$ und $x \in \mathbb{R}$ mit $|x| < 1$. Untersuchen Sie die folgende Binomialreihe auf Konvergenz und absolute Konvergenz:
	      \[
		      B_a(x) := \sum_{n=0}^{\infty} \binom{a}{n}x^n.
	      \]
	\item Zeigen Sie für die Binomialreihe aus a), dass für alle $x \in \mathbb{R}$ mit $|x| < 1$ und alle $a, b \in \mathbb{R}$ die folgende Funktionalgleichung gilt:
	      \[
		      B_a(x) \cdot B_b(x) = B_{a+b}(x).
	      \]
\end{enumerate}
\end{problem}

\begin{problem}[Sinus und Cosinus hyperbolicus]{4 Punkte}
Beweisen Sie die folgenden Funktionalgleichungen für die Grenzwerte der in der Vorlesung definierten unendlichen Reihen von Sinus hyperbolicus und Cosinus hyperbolicus.
\begin{enumerate}
	\item $\cosh(x + y) = \cosh(x) \cosh(y) + \sinh(x) \sinh(y)$,
	\item $\sinh(x + y) = \cosh(x) \sinh(y) + \sinh(x) \cosh(y)$,
	\item $(\cosh(x))^2 - (\sinh(x))^2 = 1$.
\end{enumerate}
Zusatzfrage: In welchem Sinne gelten diese Funktionalgleichungen auch für die formalen Reihen (als Folgen von Partialsummen)?
\end{problem}

\begin{problem}[Umordnungen der alternierenden Reihe]{4 Sonderpunkte}
Betrachten Sie die alternierende harmonische Reihe, die gegen $a \in \mathbb{R}$ konvergiert:
\[
	\sum_{n=1}^{\infty} (-1)^{n-1}\frac{1}{n}.
\]
\begin{enumerate}
	\item Finden Sie eine Umordnung dieser Reihe, so dass diese gegen $\frac{3}{2}a$ konvergiert.
	\item Finden Sie eine Umordnung, so dass diese bestimmt gegen $-\infty$ divergiert.
	\item Finden Sie eine Umordnung, so dass diese weder konvergiert noch bestimmt gegen $\pm\infty$ divergiert.
\end{enumerate}
\end{problem}

\end{document}
